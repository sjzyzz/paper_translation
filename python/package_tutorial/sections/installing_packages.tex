\documentclass[../package_tutorial.tex]{subfiles}

\begin{document}


这个部分包含如何安装Python包的基本知识。

很重要的一点是注意这个语境下术语“包”是用来描述一捆要安装的软件(也就是,分发的同义词)。它不代表那种你在Python源代码中导入的包(也就是,一些模块的容器)。在Python社区中,使用术语“包”来代表分发是很常见的。由于术语“分发”可能会与Linux版本或者其他大一些的软件版本例如Python自身混淆,所以使用术语“分发”通常并不被青睐。

\subsection{安装包的要求}

这个章节讲述了在安装其他Python包之前的步骤。

\subsubsection{确保你可以在命令行运行Python}

在你继续之前,确保在你的命令行中有Python并且预期的版本是可用的。你可以通过运行如下命令检查:

\begin{lstlisting}
    python3 --version
\end{lstlisting}

你应该得到类似Python 3.6.3的输出。如果你没有Python,请从python.org下载最新的3.x版本或者参考Hitchhiker的Python指导中的安装Python章节。

\subsubsection{确保你可以在命令行运行pip}

额外的,你需要确保pip可用。你可以通过云心如下命令检查:

\begin{lstlisting}
    python3 -m pip --version
\end{lstlisting}

如果你从源安装Python,通过python.org的安装器或者通过Homebrew,你应该已经有pip。如果你是在Linux上并且使用你的操作系统包管理器来安装,你可能需要单独安装pip,见通过Linux包管理器安装pip/setuptools/wheel。

如果pip没有安装,那么首先尝试从标准库中牵引它:

\begin{lstlisting}
    python3 -m ensurepip --default-pip
\end{lstlisting}

如果这仍然不允许你运行python -m pip:

\begin{itemize}
    \item 安全下载get-pip.py
    \item 运行python get-pip.py。这将会下载或者升级pip。额外的,如果setuptools和wheel还没有被安装,那么它们也会被安装。
\end{itemize}

\subsubsection{确保pip,setuptools和wheel是最新版本}

尽管pip独自便足以从与构建好的二进制档案安装,最新的setuptools和wheel项目拷贝对于确保你也可以从源档案安装十分有用。

\begin{lstlisting}
    python3 -m pip install --upgrade pip setuptools wheel
\end{lstlisting}

\subsubsection{可选择的,创建一个虚拟环境}

详细信息见下面的章节,但是这里有一些典型Linux系统下的基本venv命令:

\begin{lstlisting}
    python3 -m venv tutorial_env
    source tutorial_env/bin/activate
\end{lstlisting}

这将会自tutorial\_env目录中创建一个新的虚拟环境,并将当前shell的默认python环境设置为它。

\subsection{创建虚拟环境}
\subsection{使用pip安装}

pip是推荐的安装器。接下来,我们将会覆盖最常见的使用场景。对于更多细节,见包含完整的\href{https://pip.pypa.io/en/latest/cli/}{参考指导}的\href{https://pip.pypa.io/en/latest/cli/}{pip文档}。

\subsection{从PyPI安装}

pip最常见的用法是从\href{https://packaging.python.org/glossary/#term-Python-Package-Index-PyPI}{Python包索引}中使用\href{https://packaging.python.org/glossary/#term-Requirement-Specifier}{要求指定器}安装。大致上来说,一个要求指定器有一个项目名称跟随一个可选的版本号构成。PEP 440中包含现在支持的限定符的完整规范。接下来是一些例子。

为了安装最新版本的“SomeProject”

\begin{lstlisting}
    python3 -m pip install "SomeProject"
\end{lstlisting}

为了安装特定的版本:

\begin{lstlisting}
    python3 -m pip install "SomeProject==1.4"
\end{lstlisting}

为了安装大于或等于版本1同时小鱼另外一个的版本:

\begin{lstlisting}
    python3 -m pip install "SomeProject>=1,<2"
\end{lstlisting}

为了安装兼容特定版本的版本:

\begin{lstlisting}
    python3 -m pip install "SomeProject~=1.4.2"
\end{lstlisting}

在这种情况下,这表示安装任意">=1.4.2"的"==1.4.*"的版本。

\subsection{源分发与Wheels的比较}

pip可以从sdist或者Wheels中安装,但是如果二者都存在于PyPI,pip将会倾向于兼容的wheel。你可以通过例如--no-binary选项来覆盖pip的默认行为。

Wheels是一个与sdist相比提供了更快速安装的预链接好的分发格式,特别是当项目包含编译好的扩展。

如果pip没有发现wheel,它将会在本地链接一个wheel并为将来的安装存储,而不是在将来重链接dist。

\subsection{更新包}

将已经安装的SomeProject升级至PyPI的最新版本。

\begin{lstlisting}
    python3 -m pip install --upgrade SomeProject
\end{lstlisting}

\subsection{安装至用户站点}

为了单独为当前的用户安装包,使用--user标志:

\begin{lstlisting}
    python3 -m pip install --user SomeProject
\end{lstlisting}

对于更多的信息,将pip文档中的\href{https://pip.readthedocs.io/en/latest/user_guide.html#user-installs}{用户安装}章节。

注意--user标志在虚拟环境内是无效的——所有的安装安装命令将会影响虚拟环境。

如果SomeProject定义了命令行脚本或者控制台入口点,--user将会导致它们被安装在\href{https://docs.python.org/3/library/site.html#site.USER_BASE}{用户基础}的二进制目录,它可能会也可能没有已经在你的shell的PATH中。(从版本10开始,当安装任意脚本到非PATH目录时,pip会显示警告。)如果安装后脚本在你的shell中不可用,你需要将目录添加到你的PATH中。

\begin{itemize}
    \item 在Linux和macOS中你可以通过运行python -m site --user-base来找到用户基础二进制目录。例如,这通常会打印~/.local(其中~将扩展为你的home目录的绝对路径)所以你需要在PATH中添加~/.local/bin。你可以通过\href{https://stackoverflow.com/a/14638025}{修改~/.profile}永久修改PATH。
\end{itemize}

\subsection{要求文件}

安装要求文件中指定的要求列表。

\begin{lstlisting}
    python3 -m pip install -r requirements.txt
\end{lstlisting}

\subsection{从VCS安装}

在可编辑模式下从VCS安装项目。对于详细的语法,见pip的\href{https://pip.pypa.io/en/latest/cli/pip_install/#vcs-support}{VCS支持}章节。

\subsection{从其他索引安装}

从一个备选索引下载

\begin{lstlisting}
    python3 -m pip install --index-url https://my.package.repo/simple/ SomeProject
\end{lstlisting}

在PyPI之外,在安装过程中从一个额外的索引搜索

\begin{lstlisting}
    python3 -m pip install --extra-index-url https://my.package.repo/simple/ SomeProject
\end{lstlisting}

\subsection{从本地src树安装}

在开发模式下从本地src安装,也就是,在这样一种方式下项目似乎被安装了,但是仍然可以从src树编辑。

\begin{lstlisting}
    python3 -m pip install -e <path>
\end{lstlisting}

你也可以从src正常安装

\begin{lstlisting}
    python3 -m pip install <path>
\end{lstlisting}

\subsection{从本地档案安装}

安装一个特定的源档案文件

\begin{lstlisting}
    python3 -m pip install ./downloads/SomeProject-1.0.4.tar.gz
\end{lstlisting}

从包含档案的本地目录安装(同时不检查PyPI)

\begin{lstlisting}
    python3 -m pip install --no-index --find-links=file:///local/dir/ SomeProject
    python3 -m pip install --no-index --find-links=/local/dir/ SomeProject
    python3 -m pip install --no-index --find-links=relative/dir/ SomeProject
\end{lstlisting}

\subsection{从其他源安装}

为了从其他数据源下载(例如亚马逊S3存储),你可以创建一个提供PEP503索引格式数据的助手函数,并使用--extra-index-url标志来告诉pip使用这个索引

\begin{lstlisting}
    ./s3helper --port=7777
    python3 -m pip install --extra-index-url http://localhost:7777 SomeProject
\end{lstlisting}

\subsection{下载预发布版本}

除了稳定版本,找到预发布和开发者版本。默认的,pip只寻找稳定版本。

\begin{lstlisting}
    python3 -m pip install --pre SomeProject
\end{lstlisting}

\subsection{安装Setuptools“额外”}

安装\href{https://setuptools.readthedocs.io/en/latest/userguide/dependency_management.html#optional-dependencies}{setuptools extras}

\begin{lstlisting}
    python3 -m pip install SomePackage[PDF]
    python3 -m pip install SomePackage[PDF]==3.0
    python3 -m pip install -e .[PDF]  # editable project in current directory
\end{lstlisting}

\end{document}