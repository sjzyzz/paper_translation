\documentclass{ctexart}

\usepackage{subfiles}
\usepackage{hyperref}
\usepackage{listing}

\title{The import system}
\author{杨资璋(翻译)}
\begin{document}
\maketitle

一个模块中的Python代码通过导入来访问另外一个模块中的代码。导入语句是最常用的使用导入机制方式,但是它并不是仅有的方式。例如importlib.import_module和自带的__import__()的函数也可以用来使用导入机制。

import语句结合了两个操作;它首先寻找指定名字的模块,之后它在局部作用域中将名字与搜索结果绑定。导入语句的搜索操作被定义为通过合适的参数调用__import__()函数。__import__()的返回值被用于导入语句的名字绑定操作。为了名字绑定操作的细节,请查看导入声明。

直接调用__import__()仅进行模块搜索,如果找到了,则也进行模块创建操作。虽然会有某些副作用,例如导入父包,并更新多种缓存(包括sys.modules),但是只有import语句会进行名字绑定操作。

当执行import语句时,标准的自带__import__函数会被调用。其他使用导入系统的机制(例如importlib.import_module())可能选择绕开__import__()并使用它们自己的方法来实现导入语义。

当一个模块第一次被导入是,Python会查找这个模块,如果找到了,它会创建一个模块对象,并初始化它。如果无法找到这个模块,会产生ModuleNotFoundError。当使用导入机制时,Python实现了多种寻找模块的策略。通过下面描述的多种勾子可以修改或扩展这些策略。

\section{importlib}
\subfile{sections/packaging_and_distributing_projects}
\end{document}