\documentclass{ctexart}

\usepackage{subfiles}
\usepackage{listings}
\usepackage{hyperref}
\usepackage{xcolor}

\definecolor{codegreen}{rgb}{0,0.6,0}
\definecolor{codegray}{rgb}{0.5,0.5,0.5}
\definecolor{codepurple}{rgb}{0.58,0,0.82}
\definecolor{backcolour}{rgb}{0.95,0.95,0.92}

\lstdefinestyle{mystyle}{
    backgroundcolor=\color{backcolour},   
    commentstyle=\color{codegreen},
    keywordstyle=\color{magenta},
    numberstyle=\tiny\color{codegray},
    stringstyle=\color{codepurple},
    basicstyle=\ttfamily\footnotesize,
    breakatwhitespace=false,         
    breaklines=true,                 
    captionpos=b,                    
    keepspaces=true,                 
    numbers=left,                    
    numbersep=5pt,                  
    showspaces=false,                
    showstringspaces=false,
    showtabs=false,                  
    tabsize=2
}

\lstset{style=mystyle}


\title{TORCH.UTILS.DATA}
\author{杨资璋(翻译)}

\begin{document}
\maketitle
Pytorch的数据加载工具核心是\lstinline{torch.utils.data.DataLoader}类。它提供一个遍历数据集的iterable,并支持
\begin{itemize}
    \item 映射风格和迭代风格的数据集
    \item 自定义数据加载顺序
    \item 自动形成批
    \item 单进程多进程数据加载
    \item 自动内存固定
\end{itemize}
这些选项通过DataLoader的构建函数参数设置,它的签名是:
\begin{lstlisting}[language=Python]
DataLoader(
    dataset, 
    batch_size=1, 
    shuffle=False, 
    sampler=None, 
    batch_sampler=None, 
    num_worker=0, 
    collate_fn=None,
    pin_memory=False, 
    drop_last=False, 
    timeout=0, 
    worker_init_fn=None,
    *, 
    prefetch_factor=2, 
    persistent_workers=False
)
\end{lstlisting}
下面的章节详细描述这些选项的功能和用法。
\section{数据集类型}
\subfile{sections/datasetType}
\section{数据加载顺序和Sampler}
\subfile{sections/dataLoadingOrderAndSampler}
\section{加载批和非批数据}
\subfile{sections/loadingBatchedAndNonBatchedData}
\section{单线程和多线程数据加载}
\subfile{sections/singleAndMultiProcessDataLoading}
\section{内存锁定}
\subfile{sections/memoryPinning}
\end{document}