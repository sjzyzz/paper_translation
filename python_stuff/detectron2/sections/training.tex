\documentclass[../main.tex]{subfile}

\begin{document}

根据以前的指南,你现在可以已经有一个自定义的模型和数据加载器了。为了训练,用户通常会有下面两种表现之一:

\subsection{自定义训练循环}

有了模型和数据加载器,为了写一个训练循环,剩余的其他所有部分都可以在PyTorch中找到,同时你也可以自己写一个训练循环。这种风格允许研究者更加清晰地管理并完全控制整个训练过程。一个例子是\href{https://github.com/facebookresearch/detectron2/blob/master/tools/plain_train_net.py}{tools/plain\_train\_net.py}。

用户可以轻松控制任何自定义的训练逻辑。

\subsection{trainer摘要}

我们也提供了一个有帮助简化标准训练行为的勾子系统的标准“trainer”摘要。它包括下面两个实例:

\begin{itemize}
    \item \href{https://detectron2.readthedocs.io/en/latest/modules/engine.html#detectron2.engine.SimpleTrainer}{SimpleTrainer}为单一损失、单一优化器、单一源训练提供了一个最简化版本的训练循环。其他任务(checkpointing,logging等等)可以通过\href{https://detectron2.readthedocs.io/en/latest/modules/engine.html#detectron2.engine.HookBase}{勾子系统}实现。
    \item \href{https://detectron2.readthedocs.io/en/latest/modules/engine.html#detectron2.engine.defaults.DefaultTrainer}{DefaultTrainer}是一个在\href{https://github.com/facebookresearch/detectron2/blob/master/tools/train_net.py}{tools/train\_net.py}和其他脚本中使用的,由yacs设置初始化的\lstinline{SimpleTrainer}。它包括更多用户想要选择加入的标准操作,包括优化器的默认设置、学习率调整、logging、评估和checkpinging等。
\end{itemize}

为了自定义一个\lstinline{DefaultTrainer}:
\begin{enumerate}
    \item 对于简单自定义(例如,改变优化器、评测器、学习率调整器和数据加载器等),正如\href{https://github.com/facebookresearch/detectron2/blob/master/tools/train_net.py}{tools/train\_net.py},在一个子类中重写\href{https://detectron2.readthedocs.io/en/latest/modules/engine.html#detectron2.engine.defaults.DefaultTrainer}{它的方法}。
    \item 对于训练中的额外任务,查看\href{}{}是否支持。
          作为一个例子,在训练中输出hello:
          \begin{lstlisting}
        class HelloHook(HookBase):
            def after_step(self):
                if self.trainer.iter % 100 == 0:
                    print(f"Hello at iteration {self.trainer.iter}!")
    \end{lstlisting}
    \item 使用trainer+hook系统意味着存在一些不被支持的非标准行为,尤其是在研究中。出于这个原因,我们有意保持trainer和hook系统最小化而不是功能强大。如果有任何不能被这个系统实现的行为,从\href{https://github.com/facebookresearch/detectron2/blob/master/tools/plain_train_net.py}{tools/plain\_train\_net.py}开始手动实现自定义训练逻辑是更容易的。
\end{enumerate}

\subsection{记录度量}

在训练中,detectron2模型和训练器将度量集中放在\href{https://detectron2.readthedocs.io/en/latest/modules/utils.html#detectron2.utils.events.EventStorage}{EventStorage}。你可以通过以下代码来访问它并记录度量:
\begin{lstlisting}
from detectron2.utils.events import get_event_storage

# inside the model:
if self.training:
    value = # compute the value from inputs
    storage = get_event_storage()
    storage.put_scalar("some_accuracy", value)
\end{lstlisting}
更多细节请查看文档。

度量之后会通过\href{https://detectron2.readthedocs.io/en/latest/modules/utils.html#module-detectron2.utils.events}{EventWriter}写到各种目的地。DefaultTrainer通过默认配置文件会使能一些\lstinline{EventWriter}。如何自定义它们请见上文。
\end{document}