\documentclass[../main.tex]{subfile}
\graphicspath{{\subfix{../images}}}
\begin{document}

在附录中,我们描述了一些实施细节:

\paragraph{减均值。}

$224×224$的训练/测试图像通常通过减去每个像素的平均值进行预处理[3]。当输入图像是任意尺寸时,固定尺寸的平均图像就不能直接适用。在ImageNet数据集中,我们将$224×224$的平均图像扭曲成所需的大小,然后减去它。在Pascal VOC 2007和Caltech101中,我们在所有的实验中使用恒定的平均数(128)。

\paragraph{池化仓的实现}

在应用网络时,我们使用以下实现方式来处理所有的仓。将$\text{conv}_5$特征图的宽度和高度(可以是全图或一个窗口)表示为$w$和$h$。对于一个有$n \times n$个仓的金字塔层级,第$\left( i, j \right)$个仓在$\left[ \lceil \frac{i-1}{n}w \rceil, \lfloor \frac{i}{n}w \rfloor \right] \times \left[ \lceil \frac{j-1}{n}h \rceil, \lfloor \frac{j}{n}h \rfloor \right]$。直觉上,如果需要四舍五入,我们在左边/顶部边界采取向下取整操作,在右边/底部边界采取向上取整操作。

\paragraph{将一个窗口映射到特征图}

在检测算法(以及对特征图的多视图测试)中,在图像域中给出了一个窗口,我们用它来裁剪已经被多次下采样的卷积特征图(例如$\text{conv}_5$)。所以我们需要在特征图上为窗口对齐。

在我们的实现中,我们将一个窗口的角点投射到特征图中的一个像素上,使得这个角点在图像域中最接近该特征图像素的感受野的中心。由于所有卷积层和池化层的填充,这种映射很复杂。为了简化实施,在部署过程中,我们为卷积核大小为$p$的层填充了$\lceil p/2 \rceil$个像素。这样,对于一个中心位于$\left( x^\prime, y^\prime \right)$的响应,它在图像域中的有效接受域中心位于$\left( Sx^\prime, Sy^\prime \right)$,其中$S$是之前所有步长的乘积。在我们的模型中,ZF-5在$\text{conv}_5$的$S$为16,Overfeat-5/7在$\text{conv}_{5/7}$的$S$为12。对于给定图像域中的窗口,我们通过$x^\prime = \lceil x / S \rceil + 1$以及$x^\prime = \lfloor x / S \rfloor - 1$来分别对左上角和右上角的边界进行投影。如果填充不是$\lceil p/2 \rceil$,我们需要诶$x$加上适当的偏移。

\end{document}