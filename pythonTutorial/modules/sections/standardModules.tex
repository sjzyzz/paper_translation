\documentclass[../modules.tex]{subfiles}

\begin{document}
Python有一个有一些标准模块的库,它在一个单独的文档中描述,Python库参考。一些模块被构造入解释器中;这提供了使用不是语言核心然而自建操作的途径,或者为了效率或者提供接触操作系统原型例如系统调用的途径。这样的模块集合是一个设置选项,取决于底层运行的平台。例如,\lstinline{winreg}模块仅在视窗系统中提供。一个特殊的模块值得一些关注:\lstinline{sys},它自带在每一个Python解释器中。变量\lstinline{sys.ps1}和\lstinline{sys.ps2}定义了作为提示的首要和次要字符串:
\begin{lstlisting}[language=bash]
>>> import sys
>>> sys.ps1
'>>> '
>>> sys.ps2
'... '
>>> sys.ps1 = 'C>'
C>print('Yuck!')
C>
\end{lstlisting}
这两个变量尽在交互模式下的解释器中被定义。

变量\lstinline{sys.path}是决定解释器模块搜索路径的字符串列表。它首先通过从环境变量\lstinline{PYTHONPATH}中得到的默认路径初始化,如果\lstinline{PYTHONPATH}没有被设置,则通过自建的默认路径。你可以通过使用标准的列表操作修改它
\begin{lstlisting}[language=bash]
>>> import sys
>>> sys.path.append('/ufs/guido/lib/python')
\end{lstlisting}
\end{document}