\documentclass{ctexart}

\usepackage{subfiles}
\usepackage{hyperref}
\usepackage{listings}
\usepackage{xcolor}

\definecolor{codegreen}{rgb}{0,0.6,0}
\definecolor{codegray}{rgb}{0.5,0.5,0.5}
\definecolor{codepurple}{rgb}{0.58,0,0.82}
\definecolor{backcolour}{rgb}{0.95,0.95,0.92}

\lstdefinestyle{mystyle}{
    backgroundcolor=\color{backcolour},   
    commentstyle=\color{codegreen},
    keywordstyle=\color{magenta},
    numberstyle=\tiny\color{codegray},
    stringstyle=\color{codepurple},
    basicstyle=\ttfamily\footnotesize,
    breakatwhitespace=false,         
    breaklines=true,                 
    captionpos=b,                    
    keepspaces=true,                 
    numbers=left,                    
    numbersep=5pt,                  
    showspaces=false,                
    showstringspaces=false,
    showtabs=false,                  
    tabsize=2
}

\lstset{style=mystyle}

\title{模块}
\author{杨资璋(翻译)}
\begin{document}
\maketitle
如果你退出Python解释器并再一次进入,你做的定义(函数和变量)将会丢失。然而,如果你想写一个更长的程序,你最好使用一个文本编辑器来准备给解释器的输入并使用那个文件作为输入来运行。这被称为创建一个\textit{脚本}。随着你的程序变得越来越长,你可能会想将它分为多个文件以方便维护。你也可能想使用一个你在多个程序中写过的一个方便的函数——在不将它的定义复制到每一个程序的情况下。

为了支持这个,Python有一种在一个文件中定义并在脚本中或可交互的解释器实例中使用它们的方式。这样的文件被叫做\textit{模块};模块中的定义可以被\textit{导入}到其他模块或者到\textit{主}模块中。

一个模块就是一个包含Python定义和声明的文件。文件名就是模块名加上.py后缀。在一个模块中,模块的名字(作为一个字符串)可以通过全局变量\lstinline{__name__}得到。
\section{模块的更多内容}

\subfile{sections/moreOnModules}
\section{标准模块}
\subfile{sections/standardModules}
\section{dir()函数}
\subfile{sections/theDirFunction}
\section{包}
\subfile{sections/packages}
\end{document}