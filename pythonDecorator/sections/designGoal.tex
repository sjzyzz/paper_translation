\documentclass[../main.tex]{subfile}

\begin{document}
新的语法应该
\begin{itemize}
    \item 使用任意封装,包括用户定义的可调用和已存在的自建\lstinline{classmethod()}和\lstinline{staticmethod()}。这个要求也意味着修饰符语法必须支持向封装构造者传入参数
    \item 对于每个定义支持多个封装
    \item 使得发生了什么显而易见;至少它应该显然到新用户可以在写自己的代码时忽视它
    \item 作为一种“一旦被解释很容易记住”的语法
    \item 不会使得将来的扩展更加困难
    \item 键入简单;使用它的程序可能非常频繁地使用它
    \item 不会使得快速浏览代码更加困难。对于搜索所有定义、一个特定的定义或者一个函数接收的参数也应该依旧简单。
    \item 不会将例如语言敏感的编辑器和其他“玩具分析工具”等次级支持工具变得不必要的复杂
    \item 允许后来的编译器为了修饰符优化
\end{itemize}
\end{document}