\documentclass[../main.tex]{subfile}
\graphicspath{{\subfix{../images}}}
\begin{document}

现有的深度卷积神经网络(CNN)需要一个固定尺寸(如224×224)的输入图像。这一要求是 "人为的",可能会降低任意尺寸/尺度的图像或子图像的识别精度。在这项工作中,我们为网络配备了另一种池化策略,即 "空间金字塔池化",以消除上述要求。新的网络结构,称为SPP-net,可以生成一个固定长度的表示,而不考虑图像的大小/尺度。金字塔池化对物体变形也很稳健。有了这些优势,SPP-net应该在总体上改善所有基于CNN的图像分类方法。在ImageNet 2012数据集上,我们证明了尽管各种CNN架构设计各不相同,但SPP-net能够提高它们的准确度。在Pascal VOC 2007和Caltech101数据集上,SPP-net使用单一的全图像表示法而不进行微调就能达到最先进的分类结果。

SPP-net的力量在物体检测方面也很显著。使用SPP-net,我们只计算一次整张图像的特征图,然后在任意区域(子图像)汇集特征,生成固定长度的表示,用于训练检测器。这种方法避免了重复计算卷积特征。在处理测试图像时,我们的方法比R-CNN方法快24-102倍,同时在Pascal VOC 2007上取得了更好或相当的准确度。

在2014年ImageNet大规模视觉识别挑战赛(ILSVRC)中,我们的方法在所有38个团队中物体检测排名第二,图像分类排名第三。这份手稿还介绍了为这次比赛所做的改进。

\end{document}