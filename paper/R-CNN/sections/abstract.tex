\documentclass[../main.tex]{subfile}
\graphicspath{{\subfix{../images}}}
\begin{document}

在过去的几年里,在典型的PASCAL VOC数据集上测量的物体检测性能已经趋于平稳。表现最好的方法是复杂的组合系统,通常将多个低层次的图像特征与高层次的背景相结合。在本文中,我们提出了一种简单的、可扩展的检测算法,相对于VOC 2012上的最佳结果,该算法的平均精度(mAP)提高了30\%以上,达到了53.3\%。我们的方法结合了两个关键的见解:(1) 我们可以将大容量卷积神经网络(CNN)应用于自下而上的候选区域,来对物体进行定位和分割;(2) 当标记的训练数据不足时,对辅助任务进行监督预训练,然后再进行特定领域的微调,可以产生显著的性能提升。由于我们将区域候选与CNN结合起来,我们将我们的方法称为R-CNN:具有CNN特征的区域。我们还将R-CNN与OverFeat进行了比较,后者是最近提出的基于类似CNN架构的滑动窗口检测器。我们发现,在200类ILSVRC2013检测数据集上,R-CNN比OverFeat的性能高出很多。系统源代码可见\href{http://www.cs.berkeley.edu/˜rbg/rcnn}{http://www.cs.berkeley.edu/˜rbg/rcnn}。

\end{document}