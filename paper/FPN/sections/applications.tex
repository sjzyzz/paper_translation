\documentclass[../main.tex]{subfile}
\graphicspath{{\subfix{../images}}}
\begin{document}

我们的方法对于在深度卷积网络中构建特征金字塔是普遍适用的。接下来我们在生成候选框的RPN\cite{rpn}和物体检测的Fast R-CNN\cite{fastrcnn}中使用我们的方法。为了阐述我们的方法的简单性和有效性,当我们将原有系统适配到我们的特征金字塔时,我们对它们做了最小化的修改。

\subsection{为RPN使用FPN}

RPN\cite{rpn}是滑动窗口的类别不可知物体检测器。在原本的RPN设计中,小的子网络在密集的$3\times 3$滑动窗口上评估,在单一尺度的卷积特征图上,进行物体/非物体而分类和边界框回归。这是通过一个$3\times 3$卷积层跟随着两个分别为了分类和回归的兄弟$1\times 1$卷积实现的,我们将这称为\textit{头}。物体/非物体标准和边界框回归目标是在一系列被称为\textit{锚}\cite{fpn}的参考框上定义的。为了涵盖不同形状的物体,这些锚有多个实现定义好的尺寸和高宽比。

我们通过将单一尺度特征图替换为我们的FPN来适配RPN。我们为特征金字塔中的每一个层级依附一个有相同设计的头($3\times 3$卷积和两个兄弟$1\times 1$卷积)。由于头将会在所有金字塔层级上的所有位置密集滑动,所以在一个特定层级上设置多尺度锚是不必要的。取而代之,我们为每一个层级分配单一尺度的锚。正式地来说,我们分别在$\{P_2, P_3,P_4,P_5, P_6\}$上定义了面积为$\{32^2,64^2,128^2,256^2,512^2\}$的锚。正如\cite{rpn}中所讲的,我们也在每个层级使用不同高宽比的锚$\{1:2, 1:1, 2:1\}$。所以金字塔共有15个锚。

和\cite{rpn}一样,我们基于锚与ground-truth边界框的IoU来为锚分配训练标签。正式地来说,如果一个锚在所有锚中与某个ground-truth框有最大的IoU或者与某个ground-truth框IoU大于0.7,那么它将被分配正标签。如果一个锚与所有的ground-truth框IoU都小于0.3,那么它将被分配负样本。注意ground-truth框的尺度在分配它们至各个金字塔层级时并没有被显式使用;取而代之,ground-truth框与锚相关联,而锚则已经被分配至金字塔的层级。这样,我们没有为\cite{rpn}的标签分配引入新的额外规则。

我们注意到头的参数是在所有金字塔层级共享的;我们也评估了非共享的替代方案并观测到了类似的精确度。共享参数的良好性能表明了金字塔的所有层级共享类似的语义层级。这种优势与使用特征化图片金字塔类似,其中公共的分类器头可以用在使用任意尺度图片计算得到的特征上。

有了上述的适配,RPN可以使用RPN自然地训练和测试。我们在实验中详细阐述了实现细节。

\subsection{为Fast R-CNN使用FPN}

Fast R-CNN\cite{fastrcnn}是使用RoI pooling来提取特征的基于区域的物体检测器。Fast R-CNN普遍在单一尺度特征图上使用。为了配合FPN使用,我们需要为金字塔层级分配不同尺度的RoI。

我们将特征金字塔视作由图片金字塔得到。因此我们可以调整基于区域的检测器\cite{spp, fastrcnn}在图片金字塔上运行时的分配策略。正式地来说,我们按照如下公式将宽度为$w$高度为$h$(在网络的输入图像中)的RoI分配给特征金字塔的$P_k$层:
\begin{equation} \label{equ:findk}
    k = \lfloor k_0 + \log_2(\sqrt{wh}/224) \rfloor
\end{equation}
这里224是ImageNet预训练的典型尺寸,$k_0$是$w \times h=224^2$的RoI应该被映射到的目标层级。类比给予ResNet的Faster R-CNN\cite{rpn}系统,它使用$C_4$作为单一尺度特征图,所以我们将$k_0$设置为4。直觉上来说,公式\ref{equ:findk}意味着如果RoI的尺寸变小(例如,224的1/2),那么它应该被映射到更加精细的分辨率层级(例如,$k=3$)。

我们为所有层级的所有RoI依附了预测器头(在Fast R-CNN中头是类别特定的分类器以及边界框回归器)。再一次,所有的头共享参数,无论它们的层级。在\cite{resnet}中,$\text{conv}_5$层(一个9层深度子网络)被当作头部用在conv4特征之上,但是我们的方法已经利用$\text{conv}_5$来构造特征金字塔。所以与\cite{resnet}不同的是,在最终的分类和边界框回归层之前,我们简单地采用RoI pooling来提取$7\times 7$的特征,并附加两个隐藏的1024维全连接层(每一个都跟随ReLU)。由于ResNets中不存在可用的预训练全连接层,所以这些层被随机初始化。注意相比于标准的$\text{conv}_5$头,我们的两层全连接多层感知机头部更加轻量化,更加快速。

基于这些改变,我们可以在特征金字塔的基础上训练和测试Fast R-CNN。实现细节将会在实验章节给出。

\end{document}