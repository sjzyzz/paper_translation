\documentclass[../main.tex]{subfile}
\graphicspath{{\subfix{../images}}}
\begin{document}

当前最先进的物体检测系统是如下方法的变体:假设边界框,为每个框重新采样像素或特征,并应用高质量分类器。自选择性搜索[1]以来,该管道一直在检测基准上占上风,直到目前在PASCAL VOC、COCO和ILSVRC检测上的领先结果,都是基于Faster R-CNN[2],尽管有更深的特征,如 [3]。虽然准确,但这些方法对于嵌入式系统来说计算量太大,即使使用高端硬件,对于实时应用程序来说也太慢。这些方法的检测速度通常以每帧秒数 (SPF) 为单位,甚至是最快的高精度检测器 Faster R-CNN 的运行速度仅为每秒 7 帧 (FPS)。已经有许多尝试通过攻击检测管道的每个阶段来构建更快的检测器(参见第 4 节中的相关工作),但到目前为止,显着提高速度只是以显着降低检测精度为代价。

本文提出了第一个基于深度网络的物体检测器,该检测器不对假设的边界框像素或特征进行重采样,\textit{并}与重采样的方法一样准确。这使得高准确度的检测速度有了明显的提高(在VOC2007测试中,SSD达到了59 FPS,mAP为74.3\%,而Faster R-CNN为7 FPS,mAP为73.2\%或YOLO为 45 FPS,mAP 63.4\%)。速度的根本提高来自于消除了边界框候选和随后的像素或特征重采样阶段。我们不是第一个这样做的(参见[4,5]),但通过增加一系列的改进,我们方法比以前的尝试显著提高了准确性。我们的改进包括使用小卷积核来预测物体类别和边界框位置的偏移,为不同的高宽比检测使用单独的预测器(核),并将这些核应用于网络后期的多个特征图,来在多个尺度上进行检测。通过这些修改——特别是在不同尺度上使用多层预测——我们可以使用相对较低的分辨率输入实现高精确度,并进一步提高了检测速度。虽然这些贡献独立来看可能很小,但我们注意到所产生的系统将PASCAL VOC的实时检测精度从YOLO的63.4\% mAP到我们的SSD的74.3\% mAP。这比最近非常引人注目的关于残余网络\cite{resnet}的工作在检测精度上有者更大的相对改善。此外,高质量检测的速度的显著提高可以扩大计算机视觉的使用范围。

我们将我们的贡献总结如下:
\begin{itemize}
    \item 我们介绍了SSD,一种多类别的单次检测器,它比以前的单次检测器(YOLO)更快,而且明显更准确,实际上与进行显示区域候选和集合的较慢技术一样准确(包括快速R-CNN)。
    \item SSD的核心是使用应用于特征图上的小卷积核预测类别分数和固定的默认边界框的偏移量。
    \item 为了达到较高的检测精度,我们从不同尺度的特征图中产生不同尺度的预测,并显式按高宽比分开预测。
    \item 这些设计特点导致了简单的端到端训练和高精确度,甚至在低分辨率的输入图像上也是如此,进一步改善了速度与精确度的权衡。
    \item 实验包括在PASCAL VOC、COCO和ILSVRC上对不同输入尺寸的模型进行时间和精度分析,并与一系列最新的最先进的方法进行比较。
\end{itemize}

\end{document}