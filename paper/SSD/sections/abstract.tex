\documentclass[../main.tex]{subfile}
\graphicspath{{\subfix{../images}}}
\begin{document}
我们提出了一种在图片中使用单个深度神经网络进行物体检测的方法。我们的方法,名为SSD,将边界框的输出空间离散为存在于特征图每个位置的一组具有不同长宽比和尺度的默认框。在预测时,网络为每个默认框中的每个物体类别的存在生成分数,并对框进行调整以更好的匹配物体形状。此外,该网络结合了来自具有不同分辨率的多个特征图的预测,以自然地处理各种大小的物体。相对于需要物体候选的方法,SSD 很简单,因为它完全消除了候选生成和后续像素重采样或特征重采样阶段,并将所有计算封装在单个网络中。这使得SSD易于训练并可以直接明了地集成到需要检测组件的系统中。在PASCAL VOC、COCO和ILSVRC数据集上的实验结果证实,SSD与使用额外物体候选步骤的方法的准确性可以相提并论,同时SSD速度要快得多,同时为训练和推理提供统一的框架。SSD在VOC2007 \textit{text}上,以$300 \times 300$的图片作为输入,在Nvidia Titan X上以59 FPS的速度达到了74.3\%mAP的准确率,以$512 \times 512$的图片作为输入,达到了76.9\%mAP的准确率,优于同类最先进的Faster R-CNN模型。与其他单阶段方法相比,即使输入图像尺寸更小,SSD依然具有更好的准确性。代码位于:\href{https://github.com/weiliu89/caffe/tree/ssd}{https://github.com/weiliu89/caffe/tree/ssd}。
\end{document}