\documentclass[../main.tex]{subfile}
\graphicspath{{\subfix{../images}}}
\begin{document}

本节将介绍我们提出的SSD检测框架(\ref{sec:model}节)和相关的训练方法(\ref{sec:training}节)。之后,第3节将介绍了对应数据集的具体模型细节和实验结果。

\subsection{模型} \label{sec:model}

SSD方法基于一个前向传播卷积网络,它产生一个固定大小的边界框集合,并对这些框中存在的物体类别实例进行评分,然后通过一个非最大抑制步骤来产生最终的检测结果。前期的网络层是基于用于高质量图像分类的标准结构(在分类层之前截断),我们将其称之基础网络。然后,我们向网络添加辅助结构,以产生具有以下关键特征的检测结果:

\paragraph{用于检测的多尺度特征图} 我们在截断的基础网络的末端添加卷积特征层。这些层的大小逐渐减少,并允许在多个尺度上预测检测。预测检测的卷积模型对于每个特征层都是不同的(参考Overfeat[4]和YOLO[5],它们在单一尺度的特征图上进行操作)。

\paragraph{用于检测的卷积预测器}

\subsection{训练} \label{sec:training}

\end{document}