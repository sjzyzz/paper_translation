\documentclass[../main.tex]{subfile}
\graphicspath{{\subfix{../images}}}
\begin{document}
最新的物体检测网络依赖候选区域生成算法来假定物体位置。例如SPPnet\cite{spp}和Fast R-CNN\cite{fastrcnn}的先进网络已经减少了检测网络的运行时间,暴露出候选区域生成是瓶颈。在这个工作中,我们提出了与检测网络分享全图卷积特征的\textit{候选区域生成网络}(RPN),因此使得候选区域生成几乎毫无代价。RPN是在每个位置同时预测物体边界和置信度的全卷积网络。RPN通过端到端的方式训练来生成高质量的候选区域,这将会被Fast R-CNN用来检测。更进一步地,通过共享卷积特征,我们将RPN和Fast R-CNN合并为一个网络——使用最近流行的关于神经网络的术语“注意力”机制,RPN组件告诉整个网络应该看向何处。对于非常深的VGG-16\cite{vgg}模型,在每张图片使用300个候选区域的情况下,我们的检测系统在单块GPU上达到了5fps的速度,同时在PASCAL VOC 2017和COCO数据集上达到了最好的效果。在ILSVRC和COCO2015竞赛中,Faster R-CNN和RPN是多个赛道第一名的基础。代码已经开源。
\end{document}