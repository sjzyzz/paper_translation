\documentclass[../main.tex]{subfile}

\begin{document}

\lstinline{DataLoader}的构造函数中最重要的参数就是\lstinline{dataset},它表明了加载数据的来源。Pytorch支持两种不同类型的数据集:
\begin{itemize}
    \item 映射风格的数据集
    \item 迭代形式的数据集
\end{itemize}

\subsection{映射形式的数据集}

一个映射风格的数据集是实现了\lstinline{__getitem__()}和\lstinline{__len__()}协议的数据集,它提供一个从索引或者键到数据样本的映射(可能是不完整的)。

例如,这样的数据集,当使用\lstinline{dataset[idx]}访问时,可以从磁盘上的文件夹读取第\lstinline{idx}个图片和它对应的标签。

更多细节见\lstinline{Dataset}。

\subsection{迭代形式的数据集}

一个迭代形式的数据集是实现了\lstinline{__iter__()}协议的\lstinline{IterableDataset}的自类的实例,并提供遍历数据样本的iterable。这种类型的数据集尤其适合任意读取非常昂贵甚至不可行的情况,以及批大小取决于取得的数据的情况。

例如,这样的数据集,当调用\lstinline{iter(dataset)}时,将会返回一个数据集的数据读取流,一个远程的服务器甚至是记录可以实时生成。

更多细节见\lstinline{IterableDataset}。

\textcolor{blue}{NOTE}:当配合使用\lstinline{IterableDataset}和多进程数据加载时,每一个进程的数据集对象是重复的,所以必须设置备份来避免重复数据。如何实现见\lstinline{IterableDataset}文档。

\subsection{IterableDataset的文档}

一个可迭代数据集。

所有表示数据样本的iterable都应该继承它。当数据来自一个流时,这种形式的数据集尤其有用。

所有的子类应该重写\lstinline{__iter__()},它将返回一个这个数据集中样本的iterator。

当通过\lstinline{DataLoader}使用一个子类时,数据集中的每一个项都会通过\lstinline{DataLoader} iterator得到。当\lstinline{num_worker > 0}时,每个工人进程将会有数据集对象的一个不同备份,所以人们通常想独立设置每个备份来避免各个工人返回相同的数据。当在工人进程中调用\lstinline{get_worker_info()}时,它会返回关于这个工人的信息。可以在数据集的\lstinline{__iter__()}或\lstinline{DataLoader}的\lstinline{worker_init_info}选项中使用它来修改每个备份的行为。

例子1:在\lstinline{__iter__()}中修改所有工人的负担:

\begin{lstlisting}[language=Python]
>>> class MyIterableDataset(torch.utils.data.IterableDataset):
...     def __init__(self, start, end):
...         super(MyIterableDataset).__init__()
...         assert end > start, "this example code only works with end >= start"
...         self.start = start
...         self.end = end
...
...     def __iter__(self):
...         worker_info = torch.utils.data.get_worker_info()
...         if worker_info is None:  # single-process data loading, return the full iterator
...             iter_start = self.start
...             iter_end = self.end
...         else:  # in a worker process
...             # split workload
...             per_worker = int(math.ceil((self.end - self.start) / float(worker_info.num_workers)))
...             worker_id = worker_info.id
...             iter_start = self.start + worker_id * per_worker
...             iter_end = min(iter_start + per_worker, self.end)
...         return iter(range(iter_start, iter_end))
...
>>> # should give same set of data as range(3, 7), i.e., [3, 4, 5, 6].
>>> ds = MyIterableDataset(start=3, end=7)

>>> # Single-process loading
>>> print(list(torch.utils.data.DataLoader(ds, num_workers=0)))
[3, 4, 5, 6]

>>> # Mult-process loading with two worker processes
>>> # Worker 0 fetched [3, 4].  Worker 1 fetched [5, 6].
>>> print(list(torch.utils.data.DataLoader(ds, num_workers=2)))
[3, 5, 4, 6]

>>> # With even more workers
>>> print(list(torch.utils.data.DataLoader(ds, num_workers=20)))
[3, 4, 5, 6]
\end{lstlisting}

例子2:使用\lstinline{worker_init_fn}修改所有工人的负担:
\begin{lstlisting}[language=Python]
>>> class MyIterableDataset(torch.utils.data.IterableDataset):
...     def __init__(self, start, end):
...         super(MyIterableDataset).__init__()
...         assert end > start, "this example code only works with end >= start"
...         self.start = start
...         self.end = end
...
...     def __iter__(self):
...         return iter(range(self.start, self.end))
...
>>> # should give same set of data as range(3, 7), i.e., [3, 4, 5, 6].
>>> ds = MyIterableDataset(start=3, end=7)

>>> # Single-process loading
>>> print(list(torch.utils.data.DataLoader(ds, num_workers=0)))
[3, 4, 5, 6]
>>>
>>> # Directly doing multi-process loading yields duplicate data
>>> print(list(torch.utils.data.DataLoader(ds, num_workers=2)))
[3, 3, 4, 4, 5, 5, 6, 6]

>>> # Define a `worker_init_fn` that configures each dataset copy differently
>>> def worker_init_fn(worker_id):
...     worker_info = torch.utils.data.get_worker_info()
...     dataset = worker_info.dataset  # the dataset copy in this worker process
...     overall_start = dataset.start
...     overall_end = dataset.end
...     # configure the dataset to only process the split workload
...     per_worker = int(math.ceil((overall_end - overall_start) / float(worker_info.num_workers)))
...     worker_id = worker_info.id
...     dataset.start = overall_start + worker_id * per_worker
...     dataset.end = min(dataset.start + per_worker, overall_end)
...

>>> # Mult-process loading with the custom `worker_init_fn`
>>> # Worker 0 fetched [3, 4].  Worker 1 fetched [5, 6].
>>> print(list(torch.utils.data.DataLoader(ds, num_workers=2, worker_init_fn=worker_init_fn)))
[3, 5, 4, 6]

>>> # With even more workers
>>> print(list(torch.utils.data.DataLoader(ds, num_workers=20, worker_init_fn=worker_init_fn)))
[3, 4, 5, 6]
\end{lstlisting}
\end{document}