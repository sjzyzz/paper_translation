\documentclass[../main.tex]{subfile}

\begin{document}
对于迭代形式的数据集,数据加载顺序完全由用户定义的iterable控制。这使得块读取和动态批大小(也就是,每次通过得到一个批打包的样本)的实现更加容易。

这个章节的剩余部分考虑映射风格的数据集。\lstinline{torch.utils.data.Sampler}类被用来指定数据加载中使用的索引/键序列。它们表示遍历数据集索引的iterable对象。例如,在随机梯度下降的通常情形中,\lstinline{Sampler}可以随意排列索引列表并每次得到一个索引,或者为迷你批随机梯度下降得到一小部分索引。

根据传入\lstinline{DataLoader}的\lstinline{shuffle}参数,一个序列或者打乱的采样器会被自动构建。此外,用户也可以使用\lstinline{sampler}参数来指定自定义的每次得到下一个将要抓取的索引/键的\lstinline{Sampler}对象。

每次得到批索引列表的自定义\lstinline{Sampler}可以通过\lstinline{batch_sampler}传入。自动批形成也可以通过\lstinline{batch_size}和\lstinline{drop_last}参数使能。更多细节见下一个章节。

\textcolor{blue}{NOTE}:由于迭代形式的数据集没有键或索引的表示,所以\lstinline{sampler}和\lstinline{batch_sampler}都与迭代形式的数据集不兼容。

\end{document}